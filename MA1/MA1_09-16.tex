% Document setup
\documentclass[12pt]{article}
\usepackage[margin=1in]{geometry}
\usepackage{fancyhdr}
\usepackage{lastpage}

\pagestyle{fancy}
\lhead{Richard Whitehill}
\chead{PHYS 714 -- Mini-Assignment 1}
\rhead{09/16/22}
\cfoot{\thepage \hspace{1pt} of \pageref{LastPage}}

% Encoding
\usepackage[utf8]{inputenc}
\usepackage[T1]{fontenc}

% Math/Physics Packages
\usepackage{amsmath}
\usepackage{mathtools}
\usepackage{dsfont}
\usepackage[arrowdel]{physics}
\usepackage{siunitx}

\AtBeginDocument{\RenewCommandCopy\qty\SI}

% Reference Style
\usepackage{hyperref}
\hypersetup{
    colorlinks=true,
    linkcolor=blue,
    filecolor=magenta,
    urlcolor=cyan,
    citecolor=green
}

\newcommand{\eref}[1]{Eq.~(\ref{eq:#1})}
\newcommand{\erefs}[2]{Eqs.~(\ref{eq:#1})--(\ref{eq:#2})}

\newcommand{\fref}[1]{Fig.~\ref{fig:#1}}
\newcommand{\frefs}[2]{Figs.~\ref{fig:#1}--\ref{fig:#2}}

\newcommand{\tref}[1]{Table~\ref{tab:#1}}
\newcommand{\trefs}[2]{Tables~\ref{tab:#1}-\ref{tab:#2}}

% Figures and Tables 
\usepackage{graphicx}
\usepackage{float}

\newcommand{\bef}{\begin{figure}[h!]\begin{center}}
\newcommand{\eef}{\end{center}\end{figure}}

\newcommand{\bet}{\begin{table}[h!]\begin{center}}
\newcommand{\eet}{\end{center}\end{table}}

% tikz
\usepackage{tikz}
\usetikzlibrary{calc}
\usetikzlibrary{decorations.pathmorphing}
\usetikzlibrary{decorations.markings}
\usetikzlibrary{arrows.meta}
\usetikzlibrary{positioning}

% tcolorbox
\usepackage[most]{tcolorbox}
\usepackage{xcolor}
\usepackage{xifthen}
\usepackage{parskip}

\newcommand*{\eqbox}{\tcboxmath[
    enhanced,
    colback=black!10!white,
    colframe=black,
    sharp corners,
    size=fbox,
    boxsep=8pt,
    boxrule=1pt
]}

% Miscellaneous Definitions/Settings
\newcommand{\prob}[2]{\textbf{#1)} #2}

\setlength{\parskip}{\baselineskip}
\setlength{\parindent}{0pt}
\setlength{\headheight}{14.49998pt}
\addtolength{\topmargin}{-2.49998pt}

\def\id{\mathds{1}}



\begin{document}
    
\prob{1}{
Prove that 
\begin{eqnarray}
    \label{eq:BAC-CAB}
    \va*{A} \cross (\va*{B} \cross \va*{C}) = \va*{B}(\va*{A} \vdot \va*{C}) - \va*{C}(\va*{A} \vdot \va*{B})
.\end{eqnarray}
}

In the following work, we will use Einstein's summation convention, where repeated indices are summed over implicitly unless otherwise stated (which for this problem will not happen).
This will make the tedious accounting of sums and indices out front of the expressions happen automatically, making the work much cleaner and straightforward.
Note that the indices here are roman letters, so the implicit sums will run over the set $\{ 1,2,3 \} $.

Without further ado, we know that the $i^{\rm th}$ component of a cross product can be written as
\begin{eqnarray}
    \label{eq:cross-product}
    (\va*{A} \cross \va*{B})_{i} = \epsilon_{ijk} A_{j}B_{k}
,\end{eqnarray}
and the dot product between two vectors is simply
\begin{eqnarray}
    \label{eq:dot-product}
    \va*{A} \vdot \va*{B} = \delta_{ij}A_{i}B_{j} = A_{i}B_{i}
.\end{eqnarray}
Hence, we can write
\begin{eqnarray}
    \label{eq:A-x-BxC}
    [\va*{A} \cross (\va*{B} \cross \va*{C})]_{i} = \epsilon_{ijk} A_{j}(\va*{B} \cross \va*{C})_{k} = \epsilon_{ijk} A_{j} \epsilon_{k \ell m} B_{\ell} C_{m} = \epsilon_{kij}\epsilon_{k \ell m} A_{j} B_{\ell} C_{m}
,\end{eqnarray}
noting that we can permute the indices of the Levi-Civita tensor without picking up any sign difference.
We can then simplify \eref{A-x-BxC} by making use of the identity $\epsilon_{kij}\epsilon_{k \ell m} = \delta_{i \ell} \delta_{jm} - \delta_{im}\delta_{j\ell}$, which gives us that
\begin{align}
    \label{eq:simplified-cross}
    [\va*{A} \cross (\va*{B} \cross \va*{C})]_{i} &= [\delta_{i\ell}\delta_{jm} - \delta_{im}\delta_{j\ell}]A_{j}B_{\ell}C_{m} \notag \\
    &= B_{i}A_{j}C_{j} - C_{i}A_{j}B_{j} = B_{i}(\va*{A} \vdot \va*{C}) - C_{i}(\va*{A} \vdot \va*{B})
.\end{align}
Thus, we can write the whole vector equivalence as
\begin{eqnarray}
    \label{eq:result}
    \eqbox{
        \va*{A} \cross (\va*{B} \cross \va*{C}) = \va*{B}(\va*{A} \vdot \va*{C}) - \va*{C}(\va*{A} \vdot \va*{B})
    }
,\end{eqnarray}
which is what we were after.


\end{document}
