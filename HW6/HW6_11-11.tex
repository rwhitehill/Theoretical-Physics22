\def\duedate{11/11/22}
\def\HWnum{6}
% Document setup
\documentclass[12pt]{article}
\usepackage[margin=1in]{geometry}
\usepackage{fancyhdr}
\usepackage{lastpage}

\pagestyle{fancy}
\lhead{Richard Whitehill}
\chead{PHYS 714 -- HW \HWnum}
\rhead{\duedate}
\cfoot{\thepage \hspace{1pt} of \pageref{LastPage}}

% Encoding
\usepackage[utf8]{inputenc}
\usepackage[T1]{fontenc}

% Math/Physics Packages
\usepackage{amsmath}
\usepackage{mathtools}
\usepackage{dsfont}
\usepackage[arrowdel]{physics}
\usepackage{siunitx}

\AtBeginDocument{\RenewCommandCopy\qty\SI}

% Reference Style
\usepackage{hyperref}
\hypersetup{
    colorlinks=true,
    linkcolor=blue,
    filecolor=magenta,
    urlcolor=cyan,
    citecolor=green
}

\newcommand{\eref}[1]{Eq.~(\ref{eq:#1})}
\newcommand{\erefs}[2]{Eqs.~(\ref{eq:#1})--(\ref{eq:#2})}

\newcommand{\fref}[1]{Fig.~\ref{fig:#1}}
\newcommand{\frefs}[2]{Figs.~\ref{fig:#1}--\ref{fig:#2}}

\newcommand{\tref}[1]{Table~\ref{tab:#1}}
\newcommand{\trefs}[2]{Tables~\ref{tab:#1}-\ref{tab:#2}}

% Figures and Tables 
\usepackage{graphicx}
\usepackage{float}

\newcommand{\bef}{\begin{figure}[h!]\begin{center}}
\newcommand{\eef}{\end{center}\end{figure}}

\newcommand{\bet}{\begin{table}[h!]\begin{center}}
\newcommand{\eet}{\end{center}\end{table}}

% tikz
\usepackage{tikz}
\usetikzlibrary{calc}
\usetikzlibrary{decorations.pathmorphing}
\usetikzlibrary{decorations.markings}
\usetikzlibrary{arrows.meta}
\usetikzlibrary{positioning}

% tcolorbox
\usepackage[most]{tcolorbox}
\usepackage{xcolor}
\usepackage{xifthen}
\usepackage{parskip}

\newcommand*{\eqbox}{\tcboxmath[
    enhanced,
    colback=black!10!white,
    colframe=black,
    sharp corners,
    size=fbox,
    boxsep=8pt,
    boxrule=1pt
]}

% Miscellaneous Definitions/Settings
\newcommand{\prob}[2]{\textbf{#1)} #2}

\setlength{\parskip}{\baselineskip}
\setlength{\parindent}{0pt}
\setlength{\headheight}{14.49998pt}
\addtolength{\topmargin}{-2.49998pt}

\def\id{\mathds{1}}



\begin{document}
    
\prob{6.5}{
Do the integral 
\begin{eqnarray}
    \label{eq:6.5-int}
    \oint_{C} \frac{\dd{z}}{z^2-1}
\end{eqnarray}
in which the contour $C$ is counterclockwise about the circle $|z| = 2$
}

We can write 
\begin{eqnarray}
    \label{eq:int-6.5-rewrite}
    \eqbox{
    \oint_{C} \frac{\dd{z}}{(z-1)(z+1)} = 2 \pi i \Big( \frac{1}{z + 1}\Big|_{z = 1} + \frac{1}{z - 1}\Big|_{z = -1} \Big) = 0
}
.\end{eqnarray}


\prob{6.9}{
Use Cauchy's integral formula
\begin{eqnarray}
    \label{eq:Cauchy-int-formula}
    f^{(n)}(z) = \frac{n!}{2 \pi i} \oint \dd{z'} \frac{f(z')}{(z' - z)^{n+1}}
\end{eqnarray}
and Rodrigues's expression 
\begin{eqnarray}
    \label{eq:6.45}
    P_{n}(x) = \frac{1}{2^{n} n!} \Big( \dv{x} \Big)^{n} (x^2-1)^{n}
\end{eqnarray}
for Legendre's polynomial $P_{n}(x)$ to derive Schlaefli's formula
\begin{eqnarray}
    \label{eq:6.46}
    P_{n}(z) = \frac{1}{2^{n} 2 \pi i} \oint \frac{(z'^2 - 1)^{n}}{(z' - z)^{n+1}} \dd{z'}
.\end{eqnarray}
}

We can denote
\begin{eqnarray}
    \label{eq:fn-Legendre}
    f(x) = (x^2 - 1)^{n} 
.\end{eqnarray}
Then,
\begin{eqnarray}
    \label{eq:Schlaefli-formula}
    f^{(n)(z)} = \frac{n!}{2 \pi i} \oint \dd{z'} \frac{(z'^2 - 1)^{n}}{(z'-z)^{n+1}} = 2^{n} n! P_{n}(z)
.\end{eqnarray}
Hence,
\begin{eqnarray}
    \label{eq:6.46-result}
    \eqbox{
    P_{n}(z) = \frac{1}{2^{n} 2 \pi i} \oint \frac{(z'^2 - 1)^{n}}{(z' - z)^{n+1}} \dd{z'}
}
.\end{eqnarray}


\prob{6.20}{
Use a contour integral to evaluate the integral
\begin{eqnarray}
    \label{eq:6.20-int}
    I_{a} = \int_{0}^{\pi} \frac{\dd{\theta}}{a + \cos{\theta}}, \quad a > 1
.\end{eqnarray}
}

Recall that we can write $\cos{\theta} = (e^{i\theta} + e^{-i\theta})/2$.
Also note that the function $(a + \cos{\theta})^{-1}$ is even with respect to $\theta$, so we can extend the integration range to be symmetric if we divide the result by $1/2$.
Hence,
\begin{eqnarray}
    \label{eq:rewrite-int-6.20}
    I_{a} = \int_{-\pi}^{\pi} \frac{\dd{\theta}}{2a + e^{i\theta} + e^{-i\theta}} 
.\end{eqnarray}
Let us define $z = e^{i\theta}$, then $e^{-i\theta} = z^{-1}$ and $\dd{\theta} = \dd{z}/iz$.
Furthermore, the contour of integration is the circle $|z|=1$.
Thus,
\begin{eqnarray}
    \label{eq:subs-int-6.20}
    I_{a} = \int_{C} \frac{1}{2a + z + z^{-1}} \frac{\dd{z}}{iz} = -i \int_{C} \frac{\dd{z}}{z^2 + 2az + 1}
.\end{eqnarray}
The function $f(z) = (z^2 + 2az + 1)^{-1}$ has poles at $z_{\pm} = -a \pm \sqrt{a^2 - 1}$.
Observe that $|z_{-}| < -1 < |z_{+}|$, implying that the only pole enclosed is $z_{+}$.
This means that 
\begin{align}
    \label{eq:label}
    I_{a} &= -i \int_{C} \frac{\dd{z}}{(z - z_{+})(z - z_{-})} = -i (2 \pi i)\Big( \frac{1}{z - z_{-}} \Big)\Big|_{z = z_{+}} \\
    &= 2 \pi \frac{1}{(-a + \sqrt{a^2 - 1}) - (-a - \sqrt{a^2 - 1})} \\
    &= \eqbox{ \frac{\pi}{\sqrt{a^2 - 1}} }
.\end{align}


\prob{6.26}{
Show that
\begin{eqnarray}
    \label{eq:6.26-int}
    \int_{0}^{\infty} \cos{ax} \, e^{-x^2} \dd{x} = \frac{1}{2}\sqrt{\pi}e^{-a^2/4}
.\end{eqnarray}
}

In this problem we again use the same reasoning as in the previous problem to write
\begin{align}
    \label{eq:rewrite-6.26-int}
    I &= \frac{1}{4} \int_{-\infty}^{\infty} (e^{iax}e^{-x^2} + e^{-iax}e^{-x^2}) \dd{x} \notag \\
    &= \frac{1}{4} \int_{-\infty}^{\infty} (e^{-(x^2 - iax)} + e^{-(x^2 + iax)}) \dd{x} \notag \\
    &= \frac{1}{4} \int_{-\infty}^{\infty} \big[ e^{(-ia/2)^2}e^{-(x^2 - ix + (-ia/2)^2)} + e^{(ia/2)^2}e^{-(x^2 + iax + (ia/2)^2)} \big] \dd{x} \notag \\
    &= \frac{1}{4} e^{-a^2/4} \int_{-\infty}^{\infty} \big[ e^{-(x - ia/2)^2} + e^{-(x + ia/2)^2} \big] \dd{x}
.\end{align}
In example 6.23 it is proven that
\begin{eqnarray}
    \label{eq:integral-gauss}
    \int_{-\infty}^{\infty} e^{-m^2x^2} \dd{x} = \int_{-\infty}^{\infty} e^{-m^2(x + ic)^2} \dd{x}
,\end{eqnarray}
so using this result here, we find
\begin{eqnarray}
    \label{eq:result-6.26}
    I = \frac{\sqrt{\pi}}{2} e^{-a^2/4} 
,\end{eqnarray}
where the extra factor of $2$ came from having two gaussian integrals in \eref{rewrite-6.26-int}.


\prob{6.33}{
The Bessel function $J_{n}$ is given by the integral
\begin{eqnarray}
    \label{eq:6.33-int}
    J_{n}(x) = \frac{1}{2 \pi i} \oint_{C} e^{(x/2)(z - 1/z)} \frac{\dd{z}}{z^{n+1}} 
\end{eqnarray}
along a counterclockwise about the origin.
Find the generating function for these Bessel functions, that is, the function $G(x,z)$ whose Laurent series has the $J_{n}(x)$'s as coefficients
\begin{eqnarray}
    \label{eq:gen-func-Bessel-def}
    G(x,z) = \sum_{n=-\infty}^{\infty} J_{n}(x) z^{n} 
.\end{eqnarray}
}

\prob{6.34}{
Show that the Heaviside function $\theta(y) = (y + |y|)/(2|y|)$ is given by the integral
\begin{eqnarray}
    \label{eq:theta-int-def}
    \theta(y) = \frac{1}{2 \pi i} \int_{-\infty}^{\infty} e^{iyx} \frac{\dd{x}}{x - i\epsilon} 
\end{eqnarray}
in which $\epsilon$ is an infinitesimal positive number.
}






\end{document}
