\def\duedate{11/01/22}
\def\HWnum{5}
% Document setup
\documentclass[12pt]{article}
\usepackage[margin=1in]{geometry}
\usepackage{fancyhdr}
\usepackage{lastpage}

\pagestyle{fancy}
\lhead{Richard Whitehill}
\chead{PHYS 714 -- HW \HWnum}
\rhead{\duedate}
\cfoot{\thepage \hspace{1pt} of \pageref{LastPage}}

% Encoding
\usepackage[utf8]{inputenc}
\usepackage[T1]{fontenc}

% Math/Physics Packages
\usepackage{amsmath}
\usepackage{mathtools}
\usepackage{dsfont}
\usepackage[arrowdel]{physics}
\usepackage{siunitx}

\AtBeginDocument{\RenewCommandCopy\qty\SI}

% Reference Style
\usepackage{hyperref}
\hypersetup{
    colorlinks=true,
    linkcolor=blue,
    filecolor=magenta,
    urlcolor=cyan,
    citecolor=green
}

\newcommand{\eref}[1]{Eq.~(\ref{eq:#1})}
\newcommand{\erefs}[2]{Eqs.~(\ref{eq:#1})--(\ref{eq:#2})}

\newcommand{\fref}[1]{Fig.~\ref{fig:#1}}
\newcommand{\frefs}[2]{Figs.~\ref{fig:#1}--\ref{fig:#2}}

\newcommand{\tref}[1]{Table~\ref{tab:#1}}
\newcommand{\trefs}[2]{Tables~\ref{tab:#1}-\ref{tab:#2}}

% Figures and Tables 
\usepackage{graphicx}
\usepackage{float}

\newcommand{\bef}{\begin{figure}[h!]\begin{center}}
\newcommand{\eef}{\end{center}\end{figure}}

\newcommand{\bet}{\begin{table}[h!]\begin{center}}
\newcommand{\eet}{\end{center}\end{table}}

% tikz
\usepackage{tikz}
\usetikzlibrary{calc}
\usetikzlibrary{decorations.pathmorphing}
\usetikzlibrary{decorations.markings}
\usetikzlibrary{arrows.meta}
\usetikzlibrary{positioning}

% tcolorbox
\usepackage[most]{tcolorbox}
\usepackage{xcolor}
\usepackage{xifthen}
\usepackage{parskip}

\newcommand*{\eqbox}{\tcboxmath[
    enhanced,
    colback=black!10!white,
    colframe=black,
    sharp corners,
    size=fbox,
    boxsep=8pt,
    boxrule=1pt
]}

% Miscellaneous Definitions/Settings
\newcommand{\prob}[2]{\textbf{#1)} #2}

\setlength{\parskip}{\baselineskip}
\setlength{\parindent}{0pt}
\setlength{\headheight}{14.49998pt}
\addtolength{\topmargin}{-2.49998pt}

\def\id{\mathds{1}}



\begin{document}
    
\prob{4.1}{
Show that the Fourier integral formula 
\begin{eqnarray}
    \label{eq:fourier-int-formula}
    f(x) = \frac{1}{\pi} \int_{0}^{\infty} \int_{-\infty}^{\infty} f(y) \cos{k(y-x)} \dd{k}\dd{y}
,\end{eqnarray}
where $f$ is a real function, follows from 
\begin{eqnarray}
    \label{eq:4-9}
    f(x) = \int_{-\infty}^{\infty} \tilde{f}(k) e^{ikx} \frac{\dd{k}}{\sqrt{2\pi}} \quad \mbox{and} \quad \tilde{f}(k) = \int_{-\infty}^{\infty} f(x) e^{-ikx} \frac{\dd{x}}{\sqrt{2\pi}}
\end{eqnarray}
and
\begin{eqnarray}
    \label{eq:4-25}
    \tilde{f}^{*}(k) = \int_{-\infty}^{\infty} \frac{\dd{x}}{\sqrt{2\pi}} f(x) e^{ikx} = \tilde{f}(-k)
.\end{eqnarray}
}

We know that $f^{*}(x) = f(x)$ since $f$ is a real-valued function.
Equivalently, $\Im f(x) = 0$.
We can write 
\begin{eqnarray}
    \label{eq:f-equiv}
    f(x) = \frac{f(x) + f^{*}(x)}{2} 
.\end{eqnarray}
Thus,
\begin{align}
    \label{eq:f-equiv-fourier}
    f(x) &= \frac{1}{2} \Bigg( \int_{-\infty}^{\infty} \tilde{f}(k) e^{ikx} \frac{\dd{k}}{\sqrt{2\pi}} + \int_{-\infty}^{\infty} \tilde{f}^{*}(k) e^{-ikx} \frac{\dd{k}}{\sqrt{2\pi}} \Bigg) \\
         &= \frac{1}{2} \Bigg( \int_{-\infty}^{\infty} \tilde{f}(k) e^{ikx} \frac{\dd{k}}{\sqrt{2\pi}} + \int_{-\infty}^{\infty} \tilde{f}(-k) e^{-ikx} \frac{\dd{k}}{\sqrt{2\pi}} \Bigg) \\
         &= \frac{1}{2} \int_{-\infty}^{\infty} \Bigg(\Bigg[ \int_{-\infty}^{\infty} f(y) e^{-iky} \frac{\dd{y}}{\sqrt{2\pi}} \Bigg] e^{ikx} + \Bigg[ \int_{-\infty}^{\infty} f(y) e^{iky} \frac{\dd{y}}{\sqrt{2\pi}} \Bigg] e^{-ikx} \Bigg) \frac{\dd{k}}{\sqrt{2\pi}} \\ 
         &= \int_{-\infty}^{\infty} \int_{-\infty}^{\infty} f(y) \Big[ \frac{1}{2}\Big( e^{ik(x-y)} + e^{-ik(x-y)} \Big) \Big] \frac{\dd{k}\dd{y}}{2\pi} \\
         &= \int_{-\infty}^{\infty} \int_{-\infty}^{\infty} f(y) \cos{k(x-y)} \dd{k}\dd{y} \\
         &= \eqbox{\frac{1}{\pi} \int_{0}^{\infty} \int_{-\infty}^{\infty} f(y) \cos{k(x-y)} \dd{k}\dd{y}}
,\end{align}
which is the of the same form as \eref{fourier-int-formula} noting that cosine is odd in its argument.

\prob{4.4}{
By using the Fourier-transform formulas
\begin{subequations}
\begin{eqnarray}
    \label{eq:4-27}
    f(x) = \frac{2}{\pi}  \int_{0}^{\infty} \cos{kx} \dd{k} \int_{0}^{\infty} f(y) \cos{ky} \dd{y}
,\end{eqnarray}
if $f$ is both real and even, and
\begin{eqnarray}
    \label{eq:4-28}
    f(x) = \frac{2}{\pi}  \int_{0}^{\infty} \sin{kx} \dd{k} \int_{0}^{\infty} f(y) \sin{ky} \dd{y}
.\end{eqnarray}
\end{subequations}
if $f$ is both real and odd, derive the formulas 
\begin{subequations}
\begin{eqnarray}
    \label{eq:4-31}
    e^{-\beta |x|} = \frac{2}{\pi} \int_{0}^{\infty} \frac{\beta \cos{kx}}{\beta^2 + k^2} \dd{k}
\end{eqnarray}
and
\begin{eqnarray}
    \label{eq:4-32}
    \frac{x}{|x|}e^{-\beta |x|} = \frac{2}{\pi} \int_{0}^{\infty} \frac{k \sin{kx}}{\beta^2 + k^2} \dd{k}
\end{eqnarray}
\end{subequations}
for the even and odd extensions of the exponential $\exp(-\beta |x|)$.
}

Let us derive the expression for $\exp(-\beta |x|)$, which is clearly a real-valued function and even, so we use \eref{4-27}:
\begin{eqnarray}
    \label{eq:derive-4-31}
    \eqbox{
    e^{-\beta |x|} = \frac{2}{\pi} \int_{0}^{\infty} \int_{0}^{\infty} \cos{kx}\,e^{-\beta y} \cos{ky} \dd{y}\dd{k} = \int_{0}^{\infty} \frac{\beta}{\beta^2 + k^2} \cos{kx} \dd{k}
}
.\end{eqnarray}
Similarly, we notice that $x\exp(-\beta|x|)/|x|$ is an odd real-valued function, so
\begin{eqnarray}
    \label{eq:derive-4-32}
    \eqbox{
    \frac{x}{|x|} = \frac{2}{\pi}\int_{0}^{\infty} \int_{0}^{\infty} \sin{kx}\,e^{-\beta y} \sin{k y} \dd{y}\dd{k} = \frac{2}{\pi} \int_{0}^{\infty} \frac{k}{\beta^2 + k^2} \sin{kx} \dd{k}
}
.\end{eqnarray}




\prob{4.6}{
At time $t = 0$, a particle of mass $m$ is in a gaussian superposition of momentum eigenstates centered at $p = \hbar K$
\begin{eqnarray}
    \label{eq:wave-packet-gauss}
    \psi(x,0) = \mathcal{N} \int_{-\infty}^{\infty} e^{ikx} e^{-L^2(k - K)^2} \dd{k}
.\end{eqnarray}
} 

a) Shift $k$ by $K$ and do the integral.
Whare is the particle most likely to be found?

Let $\ell = k - K$ such that 
\begin{align}
    \label{eq:wave-packet-shift}
    \psi(x,0) &= \mathcal{N} \int_{-\infty}^{\infty} e^{i(\ell + K)x} e^{-L^2\ell^2} \dd{\ell} = e^{iKx} \mathcal{N} \int_{-\infty}^{\infty} e^{i \ell x} e^{-L^2 \ell^2} \dd{\ell} \\
              &= \mathcal{N} e^{iKx} \int_{-\infty}^{\infty} e^{-x^2/4L^2} e^{-L^2(\ell - ix/2L^2)^2} \dd{\ell} \\
              &= \mathcal{N} e^{iKx} e^{-x^2/4L^2} \Big( \frac{\sqrt{\pi}}{L} \Big)
.\end{align}
We can normalize this state as follows:
\begin{eqnarray}
    \label{eq:normalize-state}
    \int_{-\infty}^{\infty} |\psi|^2 \dd{x} = \mathcal{N}^2 \Big( \frac{\pi}{L^2} \Big) \int_{-\infty}^{\infty} e^{-x^2/2L^2} \dd{x} = \mathcal{N}^2 \Big( \frac{\pi}{L^2} \Big) \Big( L\sqrt{2\pi} \Big) = \frac{\sqrt{2 \pi^3}}{L} \mathcal{N}^2
.\end{eqnarray}
Thus, we require
\begin{eqnarray}
    \label{eq:normalization-const}
    \mathcal{N} = \Big( \frac{L^2}{2\pi^3} \Big)^{1/4} \Rightarrow \psi(x,0) = \frac{1}{(2\pi L^2)^{1/4}} e^{iKx} e^{-x^2/4L^2}
.\end{eqnarray}
Thus, the expected value of $x$ at $t = 0$ is
\begin{eqnarray}
    \label{eq:expval-x-t0}
    \eqbox{
    \expval{x} = \int_{-\infty}^{\infty} \psi^{*}(x,0) x \psi(x,0) \dd{x} = \frac{1}{\sqrt{2\pi}L} \int_{-\infty}^{\infty} x e^{-x^2/2L^2} \dd{x} = 0
}
\end{eqnarray}
since this is an odd function integrated over an even range.


b) At time $t$, the wave function $\psi(x,t)$ is $\psi(x,0)$ but with $ikx$ replaced with $ikx - i\hbar k^2t/2m$.
Shift $k$ by $K$ and do the integral.
Where is the particle most likely to be found?

The wavefunction at an arbitrary time $t > 0$ is given by
\begin{eqnarray}
    \label{eq:psi-t}
    \psi(x,t) = \mathcal{N} \int_{-\infty}^{\infty} e^{ikx - i\hbar k^2t/2m} e^{-L^2(k-K)^2} \dd{k}
.\end{eqnarray}
Again, we let $\ell = k - K$ such that
\begin{align}
    \label{eq:shift-wavefunc-t}
    \psi(x,t) &= \mathcal{N} \int_{-\infty}^{\infty} e^{i(\ell + K)x - i \hbar (\ell + K)^2t/2m} e^{-L^2\ell^2} \dd{k} \\
              &= \mathcal{N} e^{iKx}e^{-i\hbar K^2 t/2m} \int_{-\infty}^{\infty} e^{i \ell x}e^{-i\hbar \ell^2 t/2m} e^{-i\hbar \ell K t/m} e^{-L^2\ell^2} \dd{\ell} \\
              &= \mathcal{N} e^{iKx - i\hbar K^2 t/2m} \int_{-\infty}^{\infty} e^{i(x-\hbar K t/m)\ell} e^{-(L^2 + i\hbar t/2m)\ell^2} \dd{\ell} \\
              &= \mathcal{N} e^{iKx - i\hbar K^2 t/m} \sqrt{\frac{\pi}{L^2 + i\hbar t/2m}} e^{-(x-\hbar K t/2m)^2/4(L^2 + i\hbar t/2m)}
.\end{align}
Thus, we can normalize the wavefunction as follows:
\begin{gather}
    \label{eq:normalize-t}
    \mathcal{N}^2 \frac{\pi}{\sqrt{L^4 + \hbar^2 t^2/4m^2}} \int_{-\infty}^{\infty} e^{-L^2(x-\hbar K t/m)^2/2(L^4 + \hbar^2 t^2/4m^2)} \dd{x} = 1 \\
    \mathcal{N}^2 \frac{\pi}{\sqrt{L^{4} + \hbar^2t^2/4m^2}} \sqrt{\frac{2\pi\Big( L^{4} + \hbar^2 t^2/4m^2 \Big)}{L^2}} = \mathcal{N}^2 \Big[ \frac{2 \pi^3}{L^2} \Big]^{1/2} = 1 \\
    \mathcal{N} = \Big( \frac{L^2}{2\pi^3} \Big)^{1/4}
,\end{gather}
which is the same normalization as for $\psi(x,0)$ (this is expected -- actually it is required for conservation of probability).
The wavefunction for all times is then given by
\begin{eqnarray}
    \label{eq:wavefunc-t}
    \psi(x,t) = \Big( \frac{L^2}{2\pi^3} \Big)^{1/4} \sqrt{\frac{\pi}{L^2 + i\hbar t/2m}} e^{iKx - i\hbar K^2 t/2m} e^{-(x - \hbar Kt/m)^2/4(L^2 + i\hbar t/2m)}
.\end{eqnarray}
Hence, the expected position is 
\begin{align}
    \label{eq:expval-x-t}
    \expval{x} &= \frac{L}{\pi\sqrt{2\pi}} \frac{\pi}{\sqrt{L^{4} + \hbar^2 t^2/4m^2}} \int_{-\infty}^{\infty} x e^{-L^2(x-\hbar K t/m)^2/2(L^4 + \hbar^2 t^2/4m^2)} \dd{x} \\
               &= \frac{L}{\sqrt{2\pi(L^{4} + \hbar^2 t^2/4m^2)}} \frac{\hbar K t}{m} \sqrt{\frac{2\pi(L^{4} + \hbar^2t^2/4m^2)}{L^2}} \\
               &= \eqbox{ \frac{\hbar K}{m} t }
.\end{align}
This is a very nice result, appearing as the classical relation between a particle's position and momentum ($p = \hbar K$).

c) Does the wave packet spread out like $t$ or like $\sqrt{t}$ as in classical diffusion?

The spread of the gaussian wave packet is given by its standard deviation, where the time-dependent behavior goes like $t$ for large time scales.

\prob{4.11}{
Derive 
\begin{eqnarray}
    \label{eq:4-115}
    \grad \cross \va*{B} = \grad \cross (\grad \cross \va*{A}) = \grad(\grad \vdot \va*{A}) - \grad^2 \va*{A} = - \grad^2 \va*{A} = \mu_0 \va*{J}
\end{eqnarray}
from $\va*{B} = \grad \cross \va*{A}$ and Amp\`{e}re's Law $\grad \cross \va*{B} = \mu_0 \va*{J}$.
}

Observe that
\begin{eqnarray}
    \label{eq:Lap-for-A}
    \grad \cross \va*{B} = \grad \cross (\grad \cross \va*{A}) = \grad(\grad \vdot \va*{A}) - \grad^2\va*{A} = \mu_0 \va*{J}
.\end{eqnarray}
Note that we can make $\grad \vdot \va*{A} = 0$ by a suitable gauge transformation.
Thus, in the appropriate gauge,
\begin{eqnarray}
    \label{eq:Lap-for-A-1}
    \eqbox{
    -\grad^2 \va*{A} = \mu_0 \va*{J}
    } 
.\end{eqnarray}


\prob{4.14}{
Use the Green's function relations
\begin{subequations}
\begin{eqnarray}
    \label{eq:4-110}
    -\grad^2G(\va*{x} - \va*{x}') = \int \frac{\dd[3]{\va*{k}}}{(2\pi)^{3}} e^{i \va*{k} \vdot (\va*{x} - \va*{x}')} = \delta^{(3)}(\va*{x} - \va*{x}')
\end{eqnarray}
and
\begin{eqnarray}
    \label{eq:4-111}
    G(\va*{x} - \va*{x}') = \frac{1}{2\pi^2|\va*{x} - \va*{x}'|}\int_{0}^{\infty} \frac{\sin{k} \dd{k}}{k}
\end{eqnarray}
\end{subequations}
to show that
\begin{eqnarray}
    \label{eq:4-117}
    \va*{A}(\va*{x}) = \frac{\mu_0}{4\pi} \int \dd[3]\va*{x}' \frac{\va*{J}(\va*{x}')}{|\va*{x} - \va*{x}'|} 
\end{eqnarray}
satisfies \eref{4-115}.
}

Notice that the Green's function is 
\begin{eqnarray}
    \label{eq:greens-simplified}
    G(\va*{x} - \va*{x}') = \frac{1}{4\pi |\va*{x} - \va*{x}'|} 
.\end{eqnarray}
Thus,
\begin{eqnarray}
    \label{eq:Avec-int}
    \va*{A}(\va*{x}) = \mu_0 \int \dd[3]{\va*{x}'} G(\va*{x} - \va*{x}')\va*{J}(\va*{x}')
.\end{eqnarray}
Hence,
\begin{gather}
    \label{eq:lap-Avec}
    \grad^2 \va*{A}(\va*{x}) = \mu_0 \int \dd[3]{\va*{x}'} \big[ \grad^2 G(\va*{x}-\va*{x}') \big] J(\va*{x}') = \mu_0 \int \dd[3]{\va*{x}'} \delta^{(3)}(\va*{x} - \va*{x}') J(\va*{x}') \\
    \Rightarrow \eqbox{\grad^2 \va*{A}(\va*{x}) = \mu_0 \va*{J}(\va*{x})}
.\end{gather}


\prob{4.16}{
Compute the Laplace transform of $1/\sqrt{t}$ (Hint: let $t = u^2$).
}

The Laplace transform of $1/\sqrt{t}$ is computed as follows
\begin{eqnarray}
    \label{eq:lap-trans}
    \mathcal{L}\Big\{ \frac{1}{\sqrt{t}} \Big\} = \int_{0}^{\infty} \frac{1}{\sqrt{t}}e^{-st} \dd{t}
.\end{eqnarray}
If we let $t = u^2$, then $\dd{t} = 2 u \dd{u}$ and
\begin{eqnarray}
    \label{eq:lap-trans-subs}
    \eqbox{
        \mathcal{L}\Big\{ \frac{1}{\sqrt{t}} \Big\} = \int_{0}^{\infty} \frac{1}{u}e^{-su^2} 2u\dd{u} = 2 \int_{0}^{\infty} e^{-s u^2} \dd{u} = \sqrt{\frac{\pi}{s}}
}
,\end{eqnarray}
where we have assumed that $|s| > 0$ such that the integral converges.

\prob{4.17}{
Show that the commutation relations 
\begin{eqnarray}
    \label{eq:4-185}
    [a(\va*{k}),a^{\dagger}(\va*{k}')] = \delta(\va*{k} - \va*{k}') \quad \mbox{and} \quad [a(\va*{k}),a(\va*{k}')] = [a^{\dagger}(\va*{k}),a^{\dagger}(\va*{k}')] = 0
\end{eqnarray}
of the annihilation and creation operators imply the equal-time commutation relations
\begin{eqnarray}
    \label{eq:4-186}
    [\phi(\va*{x},t),\pi(\va*{y},t)]  = i\delta(\va*{x}-\va*{y}) \quad \mbox{and} \quad [\phi(\va*{x},t),\phi(\va*{y},t)] = [\pi(\va*{x},t),\pi(\va*{y},t)] = 0
\end{eqnarray}
for the field $\phi$ and its conjugate momentum $\pi$.
}

The fields
\begin{subequations}
\begin{eqnarray}
    \label{eq:phi}
    \phi(x) = \int \Big[ e^{i(\va*{k} \vdot \va*{x} - \omega_{k} t)}a(\va*{k}) + e^{-i(\va*{k} \vdot \va*{x} - \omega_{k}t)}a^{\dagger}(\va*{k}) \Big] \frac{\dd[3]{k}}{\sqrt{(2\pi)^{3}2\omega_{k}}}  
\end{eqnarray}
and
\begin{eqnarray}
    \label{eq:pi}
    \pi(x) = -i \int \Big[ e^{i(\va*{k} \vdot \va*{x} - \omega_{k} t)}a(\va*{k}) - e^{-i(\va*{k} \vdot \va*{x} - \omega_{k}t)}a^{\dagger}(\va*{k}) \Big] \sqrt{\frac{\omega_{k}}{2(2\pi^3)}}\dd[3]{k} 
.\end{eqnarray}
\end{subequations}

Hence,
\begin{align}
    \label{eq:comm-phi-pi}
    &[\phi(\va*{x},t),\pi(\va*{y},t)] = \phi(\va*{x},t)\pi(\va*{y},t) - \pi(\va*{y},t)\phi(\va*{x},t) \\
                                     &= -i \iint \Bigg[ \Big[ e^{i(\va*{k}\vdot\va*{x} - \omega_{k}t)}a(\va*{k}) + e^{-i(\va*{k}\vdot \va*{x} - \omega_{k}t)}a^{\dagger}(\va*{k}) \Big] \Big[ e^{i(\va*{k}'\vdot\va*{y} - \omega_{k'}t)}a(\va*{k}') - e^{-i(\va*{k}'\vdot \va*{y} - \omega_{k'}t)}a^{\dagger}(\va*{k}') \Big] \notag \\
                                     &-  \Big[ e^{i(\va*{k}'\vdot\va*{y} - \omega_{k'}t)}a(\va*{k}') - e^{-i(\va*{k}'\vdot \va*{y} - \omega_{k'}t)}a^{\dagger}(\va*{k}') \Big] \Big[ e^{i(\va*{k}\vdot\va*{x} - \omega_{k}t)}a(\va*{k}) + e^{-i(\va*{k}\vdot \va*{x} - \omega_{k}t)}a^{\dagger}(\va*{k}) \Big] \Bigg] \times \notag \\
                                     &\frac{1}{\sqrt{(2\pi)^3 2\omega_{k}}} \sqrt{\frac{\omega_{k'}}{2(2\pi)^3}} \dd[3]{k} \dd[3]{k'} \\
                                     &= i \iint \Bigg[ e^{i(\va*{k} \vdot \va*{x} - \omega_{k}t)}e^{-i(\va*{k}' \vdot \va*{y} - \omega_{k'}t)}\Big[a(\va*{k}),a^{\dagger}(\va*{k}')\Big] + e^{-i(\va*{k} \vdot \va*{x} - \omega_{k}t)}e^{i(\va*{k}' \vdot \va*{y} - \omega_{k'}t)} \Big[a(\va*{k}'),a^{\dagger}(\va*{k}) \Big] \Bigg] \times \notag \\
                                     &\frac{1}{2(2\pi)^3}\sqrt{\frac{\omega_{k'}}{\omega_{k}}} \dd[3]{k}\dd[3]{k'} \\
                                     &= i \int \Big[ e^{i\va*{k}(\va*{x} - \va*{y})} + e^{-i \va*{k}(\va*{x} - \va*{y})} \Big] \frac{\dd[3]{k}}{(2\pi)^3} = \frac{i}{2(2\pi)^3} 2(2\pi)^3 \delta(\va*{x} - \va*{y}) = \eqbox{ i \delta(\va*{x} - \va*{y}) }
.\end{align}

Similarly,
\begin{align}
    \label{eq:comm-phi-phi}
    &[\phi(\va*{x},t),\phi(\va*{y},t)] = \phi(\va*{x},t)\phi(\va*{y},t) - \phi(\va*{y},t)\phi(\va*{x},t) \\
                                     &= \iint \Bigg[ \Big[ e^{i(\va*{k}\vdot\va*{x} - \omega_{k}t)}a(\va*{k}) + e^{-i(\va*{k}\vdot \va*{x} - \omega_{k}t)}a^{\dagger}(\va*{k}) \Big] \Big[ e^{i(\va*{k}'\vdot\va*{y} - \omega_{k'}t)}a(\va*{k}') + e^{-i(\va*{k}'\vdot \va*{y} - \omega_{k'}t)}a^{\dagger}(\va*{k}') \Big] \notag \\
                                     &-  \Big[ e^{i(\va*{k}'\vdot\va*{y} - \omega_{k'}t)}a(\va*{k}') + e^{-i(\va*{k}'\vdot \va*{y} - \omega_{k'}t)}a^{\dagger}(\va*{k}') \Big] \Big[ e^{i(\va*{k}\vdot\va*{x} - \omega_{k}t)}a(\va*{k}) + e^{-i(\va*{k}\vdot \va*{x} - \omega_{k}t)}a^{\dagger}(\va*{k}) \Big] \Bigg] \times \notag \\
                                     &\frac{1}{\sqrt{(2\pi)^3 2\omega_{k}}} \frac{1}{\sqrt{(2\pi)^3 2\omega_{k'}}} \dd[3]{k} \dd[3]{k'} \\
                                     &= \iint \Bigg[ e^{i(\va*{k} \vdot \va*{x} - \omega_{k}t)}e^{-i(\va*{k}' \vdot \va*{y} - \omega_{k'}t)}\Big[a(\va*{k}),a^{\dagger}(\va*{k}')\Big] - e^{-i(\va*{k} \vdot \va*{x} - \omega_{k}t)}e^{i(\va*{k}' \vdot \va*{y} - \omega_{k'}t)} \Big[a(\va*{k}'),a^{\dagger}(\va*{k}) \Big] \Bigg] \times \notag \\
                                     &\frac{1}{2(2\pi)^3\sqrt{\omega_{k}\omega_{k'}}} \dd[3]{k}\dd[3]{k'} \\
                                     &= \int \Big[ e^{i \va*{k} \vdot (\va*{x} - \va*{y})} - e^{-i \va*{k} \vdot (\va*{x} - \va*{y})} \Big] \frac{\dd[3]{k}}{(2\pi)^3 \omega_{k}} = \eqbox{0}
,\end{align}
and
\begin{align}
    \label{eq:comm-pi-pi}
    &[\pi(\va*{x},t),\pi(\va*{y},t)] = \pi(\va*{x},t)\pi(\va*{y},t) - \pi(\va*{y},t)\pi(\va*{x},t) \\
                                     &= - \iint \Bigg[ \Big[ e^{i(\va*{k}\vdot\va*{x} - \omega_{k}t)}a(\va*{k}) - e^{-i(\va*{k}\vdot \va*{x} - \omega_{k}t)}a^{\dagger}(\va*{k}) \Big] \Big[ e^{i(\va*{k}'\vdot\va*{y} - \omega_{k'}t)}a(\va*{k}') - e^{-i(\va*{k}'\vdot \va*{y} - \omega_{k'}t)}a^{\dagger}(\va*{k}') \Big] \notag \\
                                     &-  \Big[ e^{i(\va*{k}'\vdot\va*{y} - \omega_{k'}t)}a(\va*{k}') - e^{-i(\va*{k}'\vdot \va*{y} - \omega_{k'}t)}a^{\dagger}(\va*{k}') \Big] \Big[ e^{i(\va*{k}\vdot\va*{x} - \omega_{k}t)}a(\va*{k}) - e^{-i(\va*{k}\vdot \va*{x} - \omega_{k}t)}a^{\dagger}(\va*{k}) \Big] \Bigg] \times \notag \\
                                     &\sqrt{\frac{\omega_{k}}{2(2\pi)^3}}\sqrt{\frac{\omega_{k'}}{2(2\pi)^3}} \dd[3]{k} \dd[3]{k'} \\
                                     &= \iint \Bigg[ e^{i(\va*{k} \vdot \va*{x} - \omega_{k}t)}e^{-i(\va*{k}' \vdot \va*{y} - \omega_{k'}t)}\Big[a(\va*{k}),a^{\dagger}(\va*{k}')\Big] - e^{-i(\va*{k} \vdot \va*{x} - \omega_{k}t)}e^{i(\va*{k}' \vdot \va*{y} - \omega_{k'}t)} \Big[a(\va*{k}'),a^{\dagger}(\va*{k}) \Big] \Bigg] \times \notag \\
                                     &\frac{\sqrt{\omega_{k}\omega_{k'}}}{2(2\pi)^3} \dd[3]{k}\dd[3]{k'} \\
                                     &= \int \Big[ e^{i \va*{k} \vdot (\va*{x} - \va*{y})} - e^{-i \va*{k} \vdot (\va*{x} - \va*{y})} \Big] \frac{\omega_{k} \dd[3]{k}}{(2\pi)^3} = \eqbox{0}
.\end{align}


\end{document}
