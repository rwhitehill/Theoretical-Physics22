\def\duedate{10/14/22}
\def\HWnum{4}
\input{../preamble_HW.tex}

\begin{document}
    
\prob{3.2}{
Show that the Fourier series 
\begin{eqnarray}
    \label{eq:3.17} 
    f(\rho,\theta) = \sum_{n=-\infty}^{\infty} \Big( \frac{\rho}{a} \Big)^{|n|} \Bigg[ \int_{0}^{2\pi} h(\theta') \frac{e^{-in\theta'}}{\sqrt{2\pi}} \dd{\theta'} \Bigg] \frac{e^{in\theta}}{\sqrt{2\pi}}
\end{eqnarray}
obeys Laplace's equation 
\begin{eqnarray}
    \label{eq:3.18} 
    \frac{1}{\rho} \pdv{\rho} \Big( \rho \pdv{f}{\rho} \Big) + \frac{1}{\rho^2} \pdv[2]{f}{\theta} = 0
\end{eqnarray}
and respects the boundary condition $f(a,\theta) = h(\theta)$.
}

$\rightarrow$ Note that Laplace's equation can be rewritten to read
\begin{eqnarray}
    \label{eq:rewrite-Laplace}
    \rho\pdv{\rho} \Big( \rho \pdv{f}{\rho} \Big) + \pdv[2]{f}{\theta} = 0
.\end{eqnarray}


For part differentiating with respect to $\theta$ we can write simply:
\begin{eqnarray}
    \label{eq:2-pdv-theta}
    \pdv[2]{f}{\theta} = -n^2f(\rho,\theta)
.\end{eqnarray}
Next, for the part with respect to $\rho$ we have
\begin{align}
    \label{eq:pdv-rho}
    \rho\pdv{f}{\rho} &= |n|f(\rho,\theta) \\
    \rho\pdv{\rho} \Big( \rho\pdv{f}{\rho} \Big) &= |n|^2 f(\rho,\theta) = n^2 f(\rho,\theta)
.\end{align}
Thus, we come to the conclusion that
\begin{eqnarray}
    \label{eq:Laplace-solution}
    \rho \pdv{\rho} \Big( \rho\pdv{f}{\rho} \Big) + \pdv[2]{f}{\theta} = n^2f(\rho,\theta) - n^2f(\rho,\theta) = 0
.\end{eqnarray}

Finally, we check the boundary condition as follows:
\begin{eqnarray}
    \label{eq:f-at-a}
    f(a,\theta) = \sum_{n=-\infty}^{\infty} \frac{e^{in\theta}}{\sqrt{2\pi}}\int_{0}^{2\pi} h(\theta') \frac{e^{-in\theta'}}{\sqrt{2\pi}} \dd{\theta'} = h(\theta)
,\end{eqnarray} 
noting that the integral is just the coefficient $h_{n}$ in the Fourier series of $h(\theta)$.


\prob{3.3}{
Find the forms that 
\begin{eqnarray}
    \label{eq:3.10}
    \begin{aligned}
        \bra{g}\ket{f} &= \sum_{n=-\infty}^{\infty} \bra{g}\ket{n}\bra{n}\ket{f} = \sum_{n=-\infty}^{\infty} g_{n}^{*}f_{n} \\
                       &= \int_{0}^{2\pi} \bra{g}\ket{x}\bra{x}\ket{f} \dd{x} = \int_{0}^{2\pi} g^{*}(x)f(x) \dd{x}
    \end{aligned} 
\end{eqnarray}
and
\begin{eqnarray}
    \label{eq:3.11}
    \begin{aligned}
        \bra{f}\ket{f} = \sum_{n=-\infty}^{\infty} |\bra{n}\ket{f}|^2 = \sum_{n=-\infty}^{\infty} |f_{n}|^2 = \int_{0}^{2\pi} |\bra{x}\ket{f}|^2 \dd{x} = \int_{0}^{2\pi} |f(x)|^2 \dd{x}
    \end{aligned} 
\end{eqnarray}
for the inner products $\bra{g}\ket{f}$ and $\bra{f}\ket{f}$ take when one uses the asymmetrical notations 
\begin{eqnarray}
    \label{eq:3.21}
    f(x) = \sum_{n=-\infty}^{\infty} d_{n} e^{inx} \quad \mbox{and} \quad d_{n} = \frac{1}{2\pi}\int_{0}^{2\pi} \dd{x} e^{-inx}f(x)
\end{eqnarray}
and
\begin{eqnarray}
    \label{eq:3.22}
    f(x) = \frac{1}{2\pi}\sum_{n=-\infty}^{\infty} c_{n}e^{inx} \quad \mbox{and} \quad c_{n} = \int_{-\pi}^{\pi} f(x) e^{-inx} \dd{x}
.\end{eqnarray}
}

For the first asymmetric convention $\displaystyle \id = \frac{1}{2\pi}\int_{0}^{2\pi} \dd{x} \ket{x}\bra{x}$ and $\displaystyle \id = \sum_{n} \ket{n}\bra{n}$, where $\bra{x}\ket{n} = e^{inx}$, so
\begin{align}
    \label{eq:first-convention}
    \bra{g}\ket{f} &= \sum_{n=-\infty}^{\infty} \bra{g}\ket{n}\bra{n}\ket{f} = \sum_{n=-\infty}^{\infty} g_{n}^{*} f_{n} \\
                   &= \frac{1}{2\pi} \int_{0}^{2\pi} \bra{g}\ket{x} \bra{x}\ket{f} \dd{x} = \frac{1}{2\pi} \int_{0}^{2\pi} g^{*}(x)f(x) \dd{x}
,\end{align}
where $f_{n}$ and $g_{n}$ are the coefficients as defined in \eref{3.21}.
Hence
\begin{align}
    \label{eq:f-f-1} 
    \bra{f}\ket{f} = \sum_{n=-\infty}^{\infty} |f_{n}|^2 = \frac{1}{2\pi} \int_{0}^{2\pi} |f(x)|^2 \dd{x}
.\end{align}

For the second asymmetric convention $\displaystyle \id = 2\pi \int_{0}^{2\pi} \dd{x} \ket{x}\bra{x}$ and $\displaystyle \id = \sum_{n} \ket{n}\bra{n}$, where $\bra{x}\ket{n} = e^{inx}/2\pi$, so
\begin{align}
    \label{eq:second-convention}
    \bra{g}\ket{f} &= \sum_{n=-\infty}^{\infty} \bra{g}\ket{n}\bra{n}\ket{f} = \sum_{n=-\infty}^{\infty} g_{n}^{*} f_{n} \\
                   &= 2\pi \int_{0}^{2\pi} \bra{g}\ket{x} \bra{x}\ket{f} \dd{x} = 2\pi \int_{0}^{2\pi} g^{*}(x)f(x) \dd{x}
,\end{align}
where $f_{n}$ and $g_{n}$ are the coefficients as defined in \eref{3.22}.
Hence
\begin{align}
    \label{eq:f-f-2} 
    \bra{f}\ket{f} = \sum_{n=-\infty}^{\infty} |f_{n}|^2 = 2\pi \int_{0}^{2\pi} |f(x)|^2 \dd{x}
.\end{align}





\prob{3.8}{}

a) Show that the Fourier series for the function $|x|$ on the interval $[-\pi,\pi]$ is 
\begin{eqnarray}
    \label{eq:|x|-fourier}
    |x| = \frac{\pi}{2} - \frac{4}{\pi}\sum_{n=0}^{\infty} \frac{\cos{(2n+1)x}}{(2n+1)^2}
.\end{eqnarray}

Since $|x|$ is a odd real-valued function we can write 
\begin{eqnarray}
    \label{eq:|x|-fourier-1}
    |x| = \frac{a_0}{2} + \sum_{n=1}^{\infty} a_{n}\cos{nx} 
,\end{eqnarray}
where
\begin{eqnarray}
    \label{eq:a0}
    a_0 = \frac{2}{\pi}\int_{0}^{\pi} x \dd{x} = \pi
\end{eqnarray}
and
\begin{eqnarray}
    \label{eq:an-bn}
    a_{n} = \frac{2}{\pi} \int_{0}^{\pi} x\cos{nx} \dd{x} = \frac{2}{\pi}\frac{\cos{(\pi n)} - 1}{n^2} = 
    \begin{cases}
        0 & n \equiv 0 ~ ({\rm mod} 2) \\
        -\frac{4}{\pi}\frac{1}{n^2} & n \equiv 1 ~ ({\rm mod} 2)
    \end{cases}
\end{eqnarray}
for $n > 1$.
Hence, 
\begin{eqnarray}
    \label{eq:|x|-fourier-2}
    \eqbox{
    |x| = \frac{\pi}{2} - \frac{4}{\pi} \sum_{n=1}^{\infty} \frac{ \cos{(2n+1)x} }{(2n+1)^2}
}
.\end{eqnarray}


b) Use this result to find a neat formula for $\pi^2/8$.

If we plug in $x = 0$, we find
\begin{eqnarray}
    \label{eq:pisq-8}
    \eqbox{
    0 = \frac{\pi}{2} - \frac{4}{\pi}\sum_{n=1}^{\infty} \frac{1}{(2n+1)^2} \Rightarrow \frac{\pi^2}{8} = \sum_{n=1}^{\infty} \frac{1}{(2n+1)^2}
}
.\end{eqnarray}



\prob{3.16}{
Suppose we wish to approximate the real square-integrable function $f(x)$ by the Fourier series with $N$ terms
\begin{eqnarray}
    \label{eq:3.171}
    f_{N}(x) = \frac{a_0}{2} + \sum_{n=1}^{N} (a_{n}\cos{nx} + b_{n}\sin{nx})
.\end{eqnarray}
Then the error
\begin{eqnarray}
    \label{eq:3.172}
    E_{N} = \int_{0}^{2\pi} \big[ f(x) - f_{N}(x) \big]^2 \dd{x}
.\end{eqnarray}
will depend upon the $2N+1$ coefficients $a_{n}$ and $b_{n}$.
The best coefficients minimize this error and satisfy the conditions
\begin{eqnarray}
    \label{eq:3.173}
    \pdv{E_{N}}{a_{n}} = \pdv{E_{N}}{b_{n}} = 0
.\end{eqnarray}
By using these conditions, find the best coefficients.
}

For the derivative of $E_{N}$ with respect to any parameter, say $\eta$, 
\begin{eqnarray}
    \label{eq:deriv-EN-param}
    \pdv{E_{N}}{\eta} = -2\int_{0}^{2\pi} \big[ f(x) - f_{N}(x) \big] \pdv{f_{N}}{\eta} \dd{x} = 0
.\end{eqnarray}
Hence,
\begin{eqnarray}
    \label{eq:deriv-params}
    \begin{aligned}
        \pdv{E_{N}}{a_0} &= -2\int_{0}^{2\pi} \frac{1}{2}\big[f(x) - f_{N}(x) \big] \dd{x} = 0 \\
        \pdv{E_{N}}{a_n} &= -2\int_{0}^{2\pi} \cos{nx}\big[f(x) - f_{N}(x) \big] \dd{x} = 0 \\
        \pdv{E_{N}}{b_n} &= -2\int_{0}^{2\pi} \sin{nx}\big[f(x) - f_{N}(x) \big] \dd{x} = 0 \\
    \end{aligned}
.\end{eqnarray}
For the following, it will be useful to recall the following properties:
\begin{align}
    \label{eq:properties}
    \int_{0}^{2\pi} \sin{(n x)} \cos{(m x)} \dd{x} &= 0 \\
    \int_{0}^{2\pi} \sin{(n x)} \sin{(m x)} \dd{x} &= \pi\delta_{nm} \\
    \int_{0}^{2\pi} \cos{(n x)} \cos{(m x)} \dd{x} &= \pi\delta_{nm}
,\end{align}
where $n,m \geq 1$.
Plugging in the definition of $f_{N}(x)$ for the first equation, we have
\begin{align}
    \label{eq:deriv-params-1}
    \int_{0}^{2\pi} f(x) \dd{x} &= \int_{0}^{2\pi} \Big( \frac{a_0}{2} + \sum_{j=1}^{N} \big[ a_{j}\cos{jx} + b_{j} \sin{jx} \big] \Big) = \pi a_0 \\
                                &\Rightarrow \eqbox{a_0 = \frac{1}{\pi} \int_{0}^{2\pi} f(x) \dd{x}}
.\end{align}
For the $a_{n}$ ($n > 1$) equations we have
\begin{align}
    \label{eq:deriv-params-2}
    \int_{0}^{2\pi} \cos{nx}f(x) \dd{x} &= \int_{0}^{2\pi} \cos{nx}\Big( \frac{a_0}{2} + \sum_{j=1}^{N} \big[ a_{j}\cos{jx} + b_{j} \sin{jx} \big] \Big) \\
                                        &= \sum_{j=1}^{N} \Bigg[ a_{j} \int_{0}^{2\pi} \cos{nx}\sin{jx} \dd{x} + b_{j} \int_{0}^{2\pi} \cos{nx}\sin{jx} \dd{x} \Bigg] \\
                                        &= \sum_{j=1}^{N} \pi a_{j} \delta_{nj} = \pi a_{n} \\
                                        &\Rightarrow \eqbox{a_{n} = \int_{0}^{2\pi} \cos{nx} f(x) \dd{x}}
.\end{align}
By a similar line of reasoning we have
\begin{eqnarray}
    \label{eq:deriv-params-3}
    \eqbox{
        b_{n} = \frac{1}{\pi} \int_{0}^{2\pi} \sin{nx} f(x) \dd{x}
} 
.\end{eqnarray}




\prob{3.19}{
    Use the commutation relation $[q,p] = i\hbar$ to show that the annihilation and creation operators satisfy the commutation relation $[a,a^{\dagger}] = 1$.
}

The creation and annihilation operators are defined as follows:
\begin{align}
    \label{eq:creation-annihilation-def}
    a &= \sqrt{\frac{m\omega}{2\hbar}}\Big( q + i \frac{p}{m\omega} \Big) \\
    a^{\dagger} &= \sqrt{\frac{m\omega}{2\hbar}}\Big( q - i \frac{p}{m\omega} \Big)
.\end{align}
Thus, the commutator
\begin{align}
    \label{eq:commutator}
    [a,a^{\dagger}] &= a a^{\dagger} - a^{\dagger} a = \frac{m\omega}{2\hbar} \Bigg[ \Big( q + i \frac{p}{m\omega} \Big) \Big( q - i \frac{p}{m\omega} \Big)  - \Big( q - i \frac{p}{m \omega} \Big) \Big( q + i \frac{p}{m \omega} \Big) \Bigg] \notag \\
                    &= \frac{m\omega}{2\hbar} \Bigg[ q^2 - \frac{i}{m \omega} qp + \frac{i}{m \omega} pq + \frac{p^2}{m^2 \omega^2} - q^2 - \frac{i}{m \omega} qp + \frac{i}{m \omega} pq - \frac{p^2}{m^2 \omega^2} \Bigg] \notag \\
                    &= \frac{1}{i \hbar} [q,p] = \eqbox{1}
.\end{align}


\prob{3.25}{}

a) Find the Fourier series for the function $f(x) = x^2$ on the interval $[-\pi,\pi]$.

Note that $x^2$ is an odd real-valued function so we can write
\begin{eqnarray}
    \label{eq:x2-fourier}
    x^2 = \frac{a_0}{2} + \sum_{n=1}^{\infty} a_{n}\cos{nx}
,\end{eqnarray}
where
\begin{eqnarray}
    \label{eq:a0_x2}
    a_0 = \frac{2}{\pi} \int_{0}^{\pi} x^2 \dd{x} = \frac{2\pi^3}{3}
\end{eqnarray}
and
\begin{eqnarray}
    \label{eq:an_x2}
    a_{n} = \frac{2}{\pi} \int_{0}^{\pi} x^2\cos{nx} \dd{x} = \frac{4 \cos{(\pi n)}}{n^2} = \frac{4}{n^2}(-1)^{n}
.\end{eqnarray}
Thus,
\begin{eqnarray}
    \label{eq:x2-fourier-1}
    x^2 = \frac{\pi^3}{3} + 4 \sum_{n=1}^{\infty} (-1)^{n}\frac{\cos{nx}}{n^2}
.\end{eqnarray}


b) Use your result at $x = \pi$ to show that 
\begin{eqnarray}
    \label{eq:3.175}
    \sum_{n=1}^{\infty} \frac{1}{n^2} = \frac{\pi^2}{6}
,\end{eqnarray}
which is the value of Riemann's zeta function $\zeta(x)$ at $x = 2$

If we plug in $x = \pi$ to our Fourier expansion above, we find
\begin{gather}
    \label{eq:x-pi-sum}
    \pi^2 = \frac{\pi^3}{3} + 4 \sum_{n=1}^{\infty} (-1)^{n} \frac{\cos{(n\pi)}}{n^2} = \frac{\pi^3}{3} + 4 \sum_{n=1}^{\infty} \frac{1}{n^2} \notag \\
    \eqbox{
    \frac{\pi^2}{6} = \sum_{n=1}^{\infty} \frac{1}{n^2}
}
.\end{gather}


\end{document}
