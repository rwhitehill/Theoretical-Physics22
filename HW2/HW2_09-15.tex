\def\duedate{09/15/2022}
\def\HWnum{2}
% Document setup
\documentclass[12pt]{article}
\usepackage[margin=1in]{geometry}
\usepackage{fancyhdr}
\usepackage{lastpage}

\pagestyle{fancy}
\lhead{Richard Whitehill}
\chead{PHYS 714 -- HW \HWnum}
\rhead{\duedate}
\cfoot{\thepage \hspace{1pt} of \pageref{LastPage}}

% Encoding
\usepackage[utf8]{inputenc}
\usepackage[T1]{fontenc}

% Math/Physics Packages
\usepackage{amsmath}
\usepackage{mathtools}
\usepackage[arrowdel]{physics}
\usepackage{siunitx}

\AtBeginDocument{\RenewCommandCopy\qty\SI}

% Reference Style
\usepackage{hyperref}
\hypersetup{
    colorlinks=true,
    linkcolor=blue,
    filecolor=magenta,
    urlcolor=cyan,
    citecolor=green
}

\newcommand{\eref}[1]{Eq.~(\ref{eq:#1})}
\newcommand{\erefs}[2]{Eqs.~(\ref{eq:#1})--(\ref{eq:#2})}

\newcommand{\fref}[1]{Fig.~\ref{fig:#1}}
\newcommand{\frefs}[2]{Figs.~\ref{fig:#1}--\ref{fig:#2}}

\newcommand{\tref}[1]{Table~\ref{tab:#1}}
\newcommand{\trefs}[2]{Tables~\ref{tab:#1}-\ref{tab:#2}}

% Figures and Tables 
\usepackage{graphicx}
\usepackage{float}

\newcommand{\bef}{\begin{figure}[h!]\begin{center}}
\newcommand{\eef}{\end{center}\end{figure}}

\newcommand{\bet}{\begin{table}[h!]\begin{center}}
\newcommand{\eet}{\end{center}\end{table}}

% tikz
\usepackage{tikz}
\usetikzlibrary{calc}
\usetikzlibrary{decorations.pathmorphing}
\usetikzlibrary{decorations.markings}
\usetikzlibrary{arrows.meta}
\usetikzlibrary{positioning}

% tcolorbox
\usepackage[most]{tcolorbox}
\usepackage{xcolor}
\usepackage{xifthen}
\usepackage{parskip}

\newcommand*{\eqbox}{\tcboxmath[
    enhanced,
    colback=black!10!white,
    colframe=black,
    sharp corners,
    size=fbox,
    boxsep=8pt,
    boxrule=1pt
]}

% Miscellaneous Definitions/Settings
\newcommand{\prob}[2]{\textbf{#1)} #2}

\setlength{\parskip}{\baselineskip}
\setlength{\parindent}{0pt}



\begin{document}
    
\prob{1.28}{
Use 
\begin{eqnarray}
    \label{eq:rotation-sq}
    (\va*{\theta} \vdot \va*{\sigma})^2 = \theta^2\id
.\end{eqnarray}
to derive the expression
\begin{eqnarray}
    \label{eq:exp-sigma}
    \exp(-i \va*{\theta} \vdot \va*{\sigma}/2) = \cos(\theta/2)\id - i \vu*{\theta} \vdot \va*{\sigma} \sin(\theta/2)
,\end{eqnarray}
for the $2 \times 2$ rotation matrix $\exp(-i \va*{\theta} \vdot \va*{\sigma}/2)$.
}

We use the Taylor series of the exponential to write
\begin{align}
    \label{eq:exp-sigma-worked}
    e^{-i \va*{\theta} \vdot \va*{\sigma}/2} &= \sum_{n=0}^{\infty} \frac{1}{n!}(-\frac{i}{2} \va*{\theta} \vdot \va*{\sigma})^{n} = \sum_{n=0}^{\infty} \frac{1}{(2n)!}(-\frac{i}{2} \va*{\theta} \vdot \va*{\sigma})^{2n} + \sum_{n=0}^{\infty} \frac{1}{(2n+1)!}(-\frac{i}{2} \va*{\theta} \vdot \va*{\sigma})^{2n+1} \notag \\
                                             &= \sum_{n=0}^{\infty} \frac{(-1)^{n}}{(2n)!} \left(\frac{\theta^2}{4} \right)^{n} - \frac{i}{2} \left( \theta \vdot \va*{\sigma} \right) \sum_{n=1}^{\infty} \frac{(-1)^{n}}{(2n+1)!}\left( \frac{\theta^2}{4} \right)^{n} \notag \\
                                             &= \sum_{n=0}^{\infty} \frac{(-1)^{n}}{(2n)!} \left( \frac{\theta}{2} \right)^{2n} - \frac{i}{2}\va*{\theta} \vdot \va*{\sigma} \sum_{n=0}^{\infty} \left[ \frac{2}{\theta}\left( \frac{\theta}{2} \right)^{2n+1} \right] \\
                                             &= \eqbox{\cos(\frac{\theta}{2})\id - i \vu*{\theta} \vdot \va*{\sigma} \sin(\frac{\theta}{2})}
.\end{align}



\prob{1.29}{
Compute the characteristic equation for the matrix $-i \va*{\theta} \vdot \va*{J}$ in which the generators are $(J_{k})_{ij} = -i\epsilon_{kij}$ is totally antisymmetric with $\epsilon_{123} = 1$.
}

Using the generator equation, we can explicitly write the matrices $J_{i}$ for $i = 1,2,3$.
\begin{eqnarray}
    \label{eq:J-matrices}
    J_{1} = -i
    \begin{pmatrix}
    0 & 0 & 0 \\
    0 & 0 & -1 \\
    0 & -1 & 0
    \end{pmatrix},
    \quad
    J_{2} = -i
    \begin{pmatrix}
    0 & 0 & -1 \\
    0 & 0 & 0 \\
    1 & 0 & 0
    \end{pmatrix},
    \quad
    J_{3} = -i
    \begin{pmatrix}
    0 & 1 & 0 \\
    -1 & 0 & 0 \\
    0 & 0 & 0
    \end{pmatrix}
.\end{eqnarray}
We can then define $\va*{\theta} = (\theta_1,\theta_2,\theta_3)$ and $\va*{J} = (J_1,J_2,J_3)$ so
\begin{align}
    \label{eq:characteristic-eq}
    {\rm det}(-i \va*{\theta} \vdot \va*{J} - \lambda \id) &= 
    \begin{vmatrix}
        -\lambda & -\theta_3 & \theta_2 \\
        \theta_3 & -\lambda & -\theta_1 \\
        -\theta_2 & \theta_1 & -\lambda
    \end{vmatrix}
    \notag \\
                                                           &=
    -\lambda(\lambda^2 + \theta_1^2) + \theta_3(-\theta_3\lambda - \theta_1\theta_2) + \theta_2(\theta_3\theta_1 - \theta_2\lambda) \notag \\
                                                           &= \eqbox{-\lambda(\lambda^2 + \theta^2)}
.\end{align}



\prob{1.30}{
Use the characteristic equation of exercise 1.29 to derive the identities
\begin{eqnarray}
    \label{eq:1-30_identities}
    \begin{aligned}
        &\exp(-i \va*{\theta} \vdot \va*{J}) = \cos{\theta}\id - i \vu*{\theta} \vdot \va*{J}\sin{\theta} + (1 - \cos{\theta})\vu*{\theta}(\vu*{\theta})^{\rm T} \\
        &\exp(-i \va*{\theta} \vdot \va*{J})_{ij} = \delta_{ij}\cos{\theta} - \sin{\theta}\epsilon_{ijk}\hat{\theta}_{k} + (1-\cos{\theta})\hat{\theta}_{i}\hat{\theta}_{j}
    \end{aligned}
.\end{eqnarray}
for the $3 \times 3$ real orthogonal matrix $\exp(-i \va*{\theta} \vdot \va*{J})$.
}

Observe a few facts.
First,
\begin{eqnarray}
    \label{eq:M-sq}
    (-i \va{\theta} \vdot \va{J})^2 = 
    \begin{pmatrix}
        \theta_2^2 + \theta_3^2 & -\theta_1\theta_2 & -\theta_1\theta_3 \\
        -\theta_1\theta_2 & \theta_1^2 + \theta_3^2 & -\theta_2\theta_3 \\
        -\theta_1\theta_3 & -\theta_2\theta_3 & \theta_1^2 + \theta_2^2
    \end{pmatrix}
    = -[\theta^2\id - \va{\theta}\va{\theta}^{\rm T}]
.\end{eqnarray}
Next, by the Cayley-Hamilton theorem we have
\begin{eqnarray}
    \label{eq:M-characteristic-solve}
    (-i \va*{\theta} \vdot \va*{J})^{3} = -(-i \va*{\theta} \vdot \va*{J})
.\end{eqnarray}
This can be extended by using the following argument.
Suppose that this result holds for any arbitrary odd integer.
That is, $(-i \va{\theta} \vdot \va{J})^{2n+1} = (-1)^{n}(-i \va{\theta} \vdot \va{J})$.
Then it is clear that, this formula holds for the next odd number
\begin{eqnarray}
    \label{eq:general-odd}
    (-i \va*{\theta} \vdot \va*{J})^{2(n+1)+1} = (-i \va*{\theta} \vdot \va*{J})^{2n+1}(-i \va*{\theta} \vdot \va*{J})^2 = (-1)^{n} (-i \va*{\theta} \vdot \va*{J})^{3} = (-1)^{n+1}(-i \va*{\theta} \vdot \va*{J})
.\end{eqnarray}
Additionally, this leads to a result for all the even integers as well:
\begin{eqnarray}
    \label{eq:general-even}
    (-i \va*{\theta} \vdot \va*{J})^{2n} = (-1)^{n-1}(-i \va*{\theta} \vdot \va*{J})^2
.\end{eqnarray}

We may now tackle \eref{1-30_identities} using the Taylor expansion of the exponential function:
\begin{align}
    \label{eq:derive-identities}
    e^{-i \va{\theta} \vdot \va{J}} &= \sum_{n=0}^{\infty} \frac{\theta^{n}}{n!}(-i \vu*{\theta} \vdot \va*{J})^{n} \notag \\
    &= \sum_{n=0}^{\infty} \frac{\theta^{2n}}{(2n)!}(-i \vu*{\theta} \vdot \va*{J})^{2n} + \sum_{n=0}^{\infty} \frac{\theta^{2n+1}}{(2n+1)!} (-i\vu*{\theta} \vdot \va*{J})^{2n+1} \notag \\
    &= \id -(-i \vu*{\theta} \vdot \va*{J})^2\sum_{n=1}^{\infty} \frac{(-1)^{n}}{(2n)!} \theta^{2n} + (-i \vu*{\theta} \vdot \va*{J}) \sum_{n=0} \frac{(-1)^{n}}{(2n+1)!} \theta^{2n+1} \notag \\
    &= \id + (\id - \vu*{\theta}\vu*{\theta}^{\rm T})(\cos{\theta} - 1) - i \vu*{\theta} \vdot \va*{J} \sin{\theta} \notag \\
    &= \eqbox{\cos{\theta}\id - i\vu*{\theta} \vdot \va*{J} \sin{\theta} + (1-\cos{\theta})\vu*{\theta}\vu*{\theta}^{\rm T}}
.\end{align}
Finally, we can translate \eref{derive-identities} into an expression relating the elements of the matrices on the right and left side of the equation as follows:
\begin{align}
    \label{eq:derive-element-identity}
    \left( e^{-i \va*{\theta} \vdot \va*{J}} \right)_{ij} &= \cos{\theta} \id_{ij} - i(\va*{\theta} \vdot \va*{J})_{ij}\sin{\theta} + (1-\cos{\theta})\left(\vu*{\theta} \vu*{\theta}^{\rm T}\right)_{ij} \notag \\
                                                          &= \delta_{ij}\cos{\theta} - i\sin{\theta}\theta_{k}(J_{k})_{ij} + (1-\cos{\theta})\hat{\theta}_{i}\hat{\theta}_{j} \notag \\
                                                          &= \eqbox{\delta_{ij}\cos{\theta} - \sin{\theta}\epsilon_{ijk}\theta_{k} + (1-\cos{\theta})\hat{\theta}_{i}\hat{\theta}_{j}}
.\end{align}




\prob{1.32}{
Consider the $2 \times 3$ matrix $A$
\begin{eqnarray}
    \label{eq:1-32_A}
    A = \begin{pmatrix}
        1 & 2 & 3 \\
        -3 & 0 & 1
    \end{pmatrix}
.\end{eqnarray}
Perform the singular value decomposition $A = USV^{\rm T}$, where $V^{\rm T}$ the transpose of $V$.
Use Matlab or another program to find the singular values and the real orthogonal matrices $U$ and $V$.

}

Notice that 
\begin{eqnarray}
    \label{eq:A-AT}
    A^{\rm T} A = V S^{\rm T} S V^{\rm T}
.\end{eqnarray}
so the eigenvalues of $A^{\rm T} A$ are the squares of the singular values, and the eigenvectors form the columns of $V$. 
Computing the eigenvalues of $A^{\rm T} A$, we solve the characteristic equation for $A^{\rm T} A$:
\begin{eqnarray}
    \label{eq:AT-A}
    \det(A^{\rm T} A - \lambda \id) = 
    \begin{vmatrix}
        10 - \lambda & 2 & 0 \\
        2  & 4 - \lambda & 6 \\
        0  & 6 & 10 - \lambda 
    \end{vmatrix}
    =
    0
    \Rightarrow \lambda = 14,~10,~0
.\end{eqnarray}
Hence, our singular values are $\sigma_1 = \sqrt{14}$ and $\sigma_2 = \sqrt{10}$.
Solving for the column vectors of $V$, we solve $(A^{\rm T}A - \lambda_i \id) v_i$ for $i = 1,~2$ and normalize the eigenvectors.
\begin{align}
    \label{eq:v1-v2}
    \lambda_1 = 14:~
    \begin{pmatrix}
    -4 & 2 & 0 \\
    2 & -10 & 6 \\
    0 & 6 & -14
    \end{pmatrix}
    v_1
    = 0
    \Rightarrow v_1 = 
    \begin{pmatrix}
    1/\sqrt{14} \\ 2/\sqrt{14} \\ 3/\sqrt{14}
    \end{pmatrix} \\
    \lambda_2 = 10:~
    \begin{pmatrix}
    0 & 2 & 0 \\
    2 & -6 & 6 \\
    0 & 6 & 0
    \end{pmatrix}
    v_2
    = 0
    \Rightarrow v_2 = 
    \begin{pmatrix}
    -3/\sqrt{10} \\ 0 \\ 1/\sqrt{10} 
    \end{pmatrix} 
.\end{align}
Finally, we find $v_3$ by simply requiring that $\{ v_1,v_2,v_3 \} $ is an orthonormal set:
\begin{eqnarray}
    \label{eq:v3}
    \begin{cases}
    \begin{aligned}
    v_1 \cdot v_3 = 0 \\
    v_2 \cdot v_3 = 0 \\
    |v_3| = 1
    \end{aligned}
    \end{cases}
    \Rightarrow 
    v_3 =
    \begin{pmatrix}
    1/\sqrt{35} \\ -5/\sqrt{35} \\ 3/\sqrt{35}
    \end{pmatrix}
.\end{eqnarray}
Now, we can determine the column vectors of $U$ by solving $u_i = \frac{1}{\sigma_{i}}Av_{i}$:
\begin{eqnarray}
    \label{eq:u1-u2}
    u_{1} = \frac{1}{\sqrt{14}}
    \begin{pmatrix}
        1 & 2 & 3 \\
        -3 & 0 & 1
    \end{pmatrix}
    \begin{pmatrix}
    1/\sqrt{14} \\ 2/\sqrt{14} \\ 3/\sqrt{14}
    \end{pmatrix}
    =
    \begin{pmatrix}
        1 \\ 0
    \end{pmatrix}
    \\
    u_{2} = \frac{1}{\sqrt{10}}
    \begin{pmatrix}
    1 & 2 & 3 \\
    -3 & 0 & 1
    \end{pmatrix}
    \begin{pmatrix}
       -3/\sqrt{10} \\ 0 \\ 3/\sqrt{10} 
    \end{pmatrix}
    =
    \begin{pmatrix}
    0 \\ 1
    \end{pmatrix}
.\end{eqnarray}
Therefore, we have our SVD matrices as
\begin{eqnarray}
    \label{eq:SVD-matrices}
    \eqbox{
    U = 
    \begin{pmatrix}
        1 & 0 \\
        0 & 1
    \end{pmatrix},
    \quad
    S =
    \begin{pmatrix}
        \sqrt{14} & 0 & 0 \\
        0 & \sqrt{10} & 0
    \end{pmatrix},
    \quad
    V = 
    \begin{pmatrix}
        1/\sqrt{14} & -3/\sqrt{10} &  1/\sqrt{35} \\
        2/\sqrt{14} & 0 & -5/\sqrt{35} \\
        3/\sqrt{14} & 1/\sqrt{10} & 3/\sqrt{35}
    \end{pmatrix}
    }       
.\end{eqnarray}
Using Wolfram, the matrices for the SVD are given as
\begin{eqnarray}
    \label{eq:SVD-matrices-wolfram}
    U = 
    \begin{pmatrix}
        1 & 0 \\
        0 & 1
    \end{pmatrix},
    \quad
    S =
    \begin{pmatrix}
        \sqrt{14} & 0 & 0 \\
        0 & \sqrt{10} & 0
    \end{pmatrix},
    \quad
    V = 
    \begin{pmatrix}
        1/\sqrt{14} & -3/\sqrt{10} &  1/\sqrt{35} \\
        \sqrt{2/7} & 0 & \sqrt{5/7} \\
        3/\sqrt{14} & 1/\sqrt{10} & 3/\sqrt{35}
    \end{pmatrix}
,\end{eqnarray}
which is the result given in \eref{SVD-matrices}.

\prob{1.35}{
    Consider the hamiltonian $H = \frac{1}{2} \hbar \omega \sigma_3$ where $\sigma_3$ is defined in (1.453).
    The entropy $S$ of this system at temperature $T$ is $S = -k{\rm Tr}[\rho \ln(\rho)]$ in which the density operator $\rho$ is
    \begin{eqnarray}
        \label{eq:dens-op}
        \rho = \frac{e^{-H/(kT)}}{{\rm Tr}\left[e^{-H/(kT)}\right]}
    .\end{eqnarray}
   Find expressions for the density operator $\rho$ and its entropy $S$. 
}

We have $\sigma_3 = \begin{pmatrix} 1 & 0 \\ 0 & - 1 \end{pmatrix}$, so
\begin{eqnarray}
    \label{eq:exp-H_kT}
    e^{-\frac{\hbar \omega \sigma_3}{kT}} = 
    \begin{pmatrix}
        e^{-\frac{\hbar \omega}{2 k T}} & 0 \\
        0 & e^{\frac{\hbar \omega}{2 k T}} 
    \end{pmatrix}
,\end{eqnarray}
so 
\begin{eqnarray}
    \label{eq:rho-mat}
    \eqbox{
    \rho = \frac{1}{e^{-\frac{\hbar \omega}{2 k T}} + e^{\frac{\hbar \omega}{2kT}}}
    \begin{pmatrix}
        e^{-\frac{\hbar \omega}{2 k T}} & 0 \\
        0 & e^{\frac{\hbar \omega}{2 k T}} 
    \end{pmatrix}
    =
    \frac{2}{\cosh(\frac{\hbar \omega}{2kT})}
    \begin{pmatrix}
        e^{-\frac{\hbar \omega}{2 k T}} & 0 \\
        0 & e^{\frac{\hbar \omega}{2 k T}} 
    \end{pmatrix}
}
.\end{eqnarray}
Additionally,
\begin{eqnarray}
    \label{eq:log-rho}
    \ln(\rho) = 
    \begin{pmatrix}
        \ln[\frac{2e^{-\hbar \omega/2kT}}{\cosh(\hbar \omega / 2kT)}] & 0 \\
        0 & \ln[\frac{2e^{\hbar \omega/2kT}}{\cosh(\hbar \omega / 2kT)}]
    \end{pmatrix}
,\end{eqnarray}
giving
\begin{align}
%    \label{eq:entropy}
    S &= -k{\rm Tr}[\rho \ln{\rho}] \notag \\
      &= -\frac{2k}{\cosh(\hbar\omega/2kT)}\left[ e^{\hbar\omega/2kT}\ln\left[\frac{2e^{\hbar\omega/2kT}}{\cosh(\hbar\omega/2kT)}\right] + e^{-\hbar\omega/2kT}\ln\left[\frac{2e^{-\hbar\omega/2kT}}{\cosh(\hbar\omega/2kT)}\right] \right] \notag \\
      &= \eqbox{ -2k \left[ \frac{\hbar\omega}{2kT}\tanh{(\hbar\omega/2kT)} +  2\ln\left[ \frac{2}{\cosh(\hbar\omega/2kT)} \right] \right]
  }
.\end{align}



\prob{1.37}{
A system that has three fermionic states has three creation operators $a_{i}^{\dagger}$ and three annihilation operators $a_{k}$ which satisfy the anticommutation relations $\{ a_{i}, a_{k}^{\dagger} \} = \delta_{ik}$ and $\{ a_{i},a_{k} \} = \{ a_{i}^{\dagger}, a_{k}^{\dagger} \} = 0$ for $i,k = 1,2,3$.
The eight states of the system are $\ket{t,u,v} \equiv (a_{1}^{\dagger})^{t}(a_{2}^{\dagger})^{u}(a_{3}^{\dagger})^{v}\ket{0,0,0}$.
We can represent them by eight 8-vectors each of which has seven 0's with a 1 in position $4t + 2u + v + 1$. 
How big should the matrices that represent the creation and annihilation operators be?
Write down the three matrices that represent the three creation operators.
}

The matrices should be $8 \times 8$ such that operating on a vector gives a vector in the same space.
For just one state, the creation operator is just
\begin{eqnarray}
    \label{eq:creation}
    a^{\dagger} = 
    \begin{pmatrix}
    0 & 0 \\
    1 & 0
    \end{pmatrix}
,\end{eqnarray}
where
\begin{eqnarray}
    \label{eq:label}
    \begin{pmatrix}
    a \\ b
    \end{pmatrix}
    =
    a \ket{0} + b\ket{1}
.\end{eqnarray}
The creation operator for the first state is the tensor product $a_{1}^{\dagger} = a^{\dagger} \otimes \id \otimes \id$ such that 
\begin{eqnarray}
    \label{eq:creation-1}
    \eqbox{
    a_{1}^{\dagger} = 
    \begin{pmatrix}
        0 & 0 & 0 & 0 & 0 & 0 & 0 & 0 \\
        0 & 0 & 0 & 0 & 0 & 0 & 0 & 0 \\
        0 & 0 & 0 & 0 & 0 & 0 & 0 & 0 \\
        0 & 0 & 0 & 0 & 0 & 0 & 0 & 0 \\
        1 & 0 & 0 & 0 & 0 & 0 & 0 & 0 \\
        0 & 1 & 0 & 0 & 0 & 0 & 0 & 0 \\
        0 & 0 & 1 & 0 & 0 & 0 & 0 & 0 \\
        0 & 0 & 0 & 1 & 0 & 0 & 0 & 0
    \end{pmatrix}
}
.\end{eqnarray}
Similarly,
\begin{eqnarray}
    \label{eq:creation-2}
    \eqbox{
    a_{2}^{\dagger} = \id \otimes a^{\dagger} \otimes  \id = 
    \begin{pmatrix}
        0 & 0 & 0 & 0 & 0 & 0 & 0 & 0 \\
        0 & 0 & 0 & 0 & 0 & 0 & 0 & 0 \\
        1 & 0 & 0 & 0 & 0 & 0 & 0 & 0 \\
        0 & 1 & 0 & 0 & 0 & 0 & 0 & 0 \\
        0 & 0 & 0 & 0 & 0 & 0 & 0 & 0 \\
        0 & 0 & 0 & 0 & 0 & 0 & 0 & 0 \\
        0 & 0 & 0 & 0 & 1 & 0 & 0 & 0 \\
        0 & 0 & 0 & 0 & 0 & 1 & 0 & 0
    \end{pmatrix}
}
\end{eqnarray}
and 
\begin{eqnarray}
    \label{eq:creation-3}
    \eqbox{
    a_{3}^{\dagger} = \id \otimes \id \otimes a^{\dagger} = 
    \begin{pmatrix}
        0 & 0 & 0 & 0 & 0 & 0 & 0 & 0 \\
        1 & 0 & 0 & 0 & 0 & 0 & 0 & 0 \\
        0 & 0 & 0 & 0 & 0 & 0 & 0 & 0 \\
        0 & 0 & 1 & 0 & 0 & 0 & 0 & 0 \\
        0 & 0 & 0 & 0 & 0 & 0 & 0 & 0 \\
        0 & 0 & 0 & 0 & 1 & 0 & 0 & 0 \\
        0 & 0 & 0 & 0 & 0 & 0 & 0 & 0 \\
        0 & 0 & 0 & 0 & 0 & 0 & 1 & 0
    \end{pmatrix}
}
.\end{eqnarray}




\end{document}
