\def\duedate{08/31/2022}
\def\HWnum{1}
% Document setup
\documentclass[12pt]{article}
\usepackage[margin=1in]{geometry}
\usepackage{fancyhdr}
\usepackage{lastpage}

\pagestyle{fancy}
\lhead{Richard Whitehill}
\chead{PHYS 714 -- HW \HWnum}
\rhead{\duedate}
\cfoot{\thepage \hspace{1pt} of \pageref{LastPage}}

% Encoding
\usepackage[utf8]{inputenc}
\usepackage[T1]{fontenc}

% Math/Physics Packages
\usepackage{amsmath}
\usepackage{mathtools}
\usepackage[arrowdel]{physics}
\usepackage{siunitx}

\AtBeginDocument{\RenewCommandCopy\qty\SI}

% Reference Style
\usepackage{hyperref}
\hypersetup{
    colorlinks=true,
    linkcolor=blue,
    filecolor=magenta,
    urlcolor=cyan,
    citecolor=green
}

\newcommand{\eref}[1]{Eq.~(\ref{eq:#1})}
\newcommand{\erefs}[2]{Eqs.~(\ref{eq:#1})--(\ref{eq:#2})}

\newcommand{\fref}[1]{Fig.~\ref{fig:#1}}
\newcommand{\frefs}[2]{Figs.~\ref{fig:#1}--\ref{fig:#2}}

\newcommand{\tref}[1]{Table~\ref{tab:#1}}
\newcommand{\trefs}[2]{Tables~\ref{tab:#1}-\ref{tab:#2}}

% Figures and Tables 
\usepackage{graphicx}
\usepackage{float}

\newcommand{\bef}{\begin{figure}[h!]\begin{center}}
\newcommand{\eef}{\end{center}\end{figure}}

\newcommand{\bet}{\begin{table}[h!]\begin{center}}
\newcommand{\eet}{\end{center}\end{table}}

% tikz
\usepackage{tikz}
\usetikzlibrary{calc}
\usetikzlibrary{decorations.pathmorphing}
\usetikzlibrary{decorations.markings}
\usetikzlibrary{arrows.meta}
\usetikzlibrary{positioning}

% tcolorbox
\usepackage[most]{tcolorbox}
\usepackage{xcolor}
\usepackage{xifthen}
\usepackage{parskip}

\newcommand*{\eqbox}{\tcboxmath[
    enhanced,
    colback=black!10!white,
    colframe=black,
    sharp corners,
    size=fbox,
    boxsep=8pt,
    boxrule=1pt
]}

% Miscellaneous Definitions/Settings
\newcommand{\prob}[2]{\textbf{#1)} #2}

\setlength{\parskip}{\baselineskip}
\setlength{\parindent}{0pt}



\begin{document}
    
\prob{1.2}{Derive the cyclicity (1.24) of the trace from Eq. (1.23).}

Using the definition of the trace, we can write
\begin{eqnarray}
    \label{eq:trace-AB}
    {\rm Tr}(AB) = \sum_{k=1}^{n} (AB)_{kk} = \sum_{k=1}^{n} \sum_{\ell=1}^{n} A_{k\ell}B_{\ell k}
.\end{eqnarray}
Note that this sum is symmetric in the indices $\{ k,\ell \} $, and since $A_{k\ell}$ and $B_{\ell k}$ are just scalars 
\begin{eqnarray}
    \label{eq:trace-BA}
    \eqbox{
    {\rm Tr}(AB) = \sum_{\ell=1}^{n} \sum_{k=1}^{n} B_{\ell k}A_{k\ell} = {\rm Tr}(BA)
    } 
.\end{eqnarray}


\prob{1.3}{Show that $\left( AB \right)^{\rm T} = B^{\rm T}A^{\rm T}$, which is Eq.~(1.26).}

Recall the definition of a matrix transpose: $(A^{\rm T})_{ij} = A_{ji}$, so
\begin{eqnarray}
    \label{eq:transpose-AB}
    (AB)^{\rm T}_{ij} = (AB)_{ji} = \sum_{k=1}^{n} A_{jk}B_{ki} = \sum_{k=1} B_{ik}^{\rm T} A_{kj}^{\rm T} = (B^{\rm T}A^{\rm T})_{ij}
.\end{eqnarray}
Hence,
\begin{eqnarray}
    \label{eq:prob3-result}
    \eqbox{
    (AB)^{\rm T} = B^{\rm T}A^{\rm T}
    }
.\end{eqnarray}



\prob{1.5}{Show that $(AB)^{\dagger} = B^{\dagger}A^{\dagger}$, which is Eq.~(1.29).}

Recall that the hermitian adjoint of a matrix is just $A^{\dagger} = (A^{\rm T})^{*}$, so
\begin{eqnarray}
    \label{eq:adjoint-AB}
    (AB)^{\dagger} = ((AB)^{\rm T})^{*} = (B^{\rm T}A^{\rm T})^{*}
.\end{eqnarray}
Observe that 
\begin{eqnarray}
    \label{eq:complex-conj-product}
    (AB)^{*}_{ij} = \left( \sum_{k=1}^{n} A_{ik}B_{kj} \right)^{*} = \sum_{k=1}^{n} A^{*}_{ik}B^{*}_{kj} = (A^{*}B^{*})_{ij}
,\end{eqnarray}
so \eref{adjoint-AB} becomes
\begin{eqnarray}
    \label{eq:adjoint-BA}
    \eqbox{
        (AB)^{\dagger} = (B^{\rm T}A^{\rm T})^{*} = (B^{\rm T})^{*}(A^{\rm T})^{*} = B^{\dagger}A^{\dagger}
    }
.\end{eqnarray}



\prob{1.7}{Show that the two $4 \times 4$ matrices (1.46) satisfy Grassman's algebra (1.11) for $n=2$.}

The matrices in Eq.~(1.46) are given by
\begin{eqnarray}
    \label{eq:matrices-1.46}
    \theta_1 = 
    \begin{pmatrix}
    0 & 0 & 1 & 0 \\
    0 & 0 & 0 & -1 \\
    0 & 0 & 0 & 0 \\
    0 & 0 & 0 & 0
    \end{pmatrix}
    \quad
    \mbox{and}
    \quad
    \theta_2 = 
    \begin{pmatrix}
        0 & 1 & 0 & 0 \\
        0 & 0 & 0 & 0 \\
        0 & 0 & 0 & 1 \\
        0 & 0 & 0 & 0
    \end{pmatrix}
.\end{eqnarray}
Using the kronecker delta, notice that we can write the components of $\theta_1$ and $\theta_2$ as follows:
\begin{eqnarray}
    \label{eq:components-th12}
    (\theta_1)_{ij} = \delta_{i 1}\delta_{j 3} - \delta_{i 2}\delta_{j 4} \\
    (\theta_2)_{ij} = \delta_{i 1}\delta_{j 2} + \delta_{i 3}\delta_{j 4}
.\end{eqnarray}

Thus,
\begin{align}
    \label{eq:th1-th2-prod}
    (\theta_1\theta_2)_{ij} &= \sum_{k=1}^{n} (\theta_1)_{ik}(\theta_2)_{kj} \\
                            &= \sum_{k=1}^{n} \delta_{i 1}\delta_{j 2}\delta_{k 3}\delta_{k 1} + \delta_{i 1}\delta_{j 4}\delta_{k 3}\delta_{k 3} - \delta_{i 2}\delta_{j 2}\delta_{k 4}\delta_{k 1} - \delta_{i 2}\delta_{j 4}\delta_{k 4}\delta_{k 3} \\
                            &= \delta_{i 1}\delta_{j 4}
.\end{align}
In the same way,
\begin{align}
    \label{eq:th2-th1-prod}
    (\theta_2\theta_1)_{ij} &= \sum_{k=1}^{n} (\theta_2)_{ik}(\theta_1)_{kj} \\
                            &= \sum_{k=1}^{n} \delta_{i 1}\delta_{j 3}\delta_{k 2}\delta_{k 1} - \delta_{i 1}\delta_{j 4}\delta_{k 2}\delta_{k 2} + \delta_{i 3}\delta_{j 3}\delta_{k 4}\delta_{k 1} - \delta_{i 3}\delta_{j 4}\delta_{k 4}\delta_{k 2} \\
                            &= -\delta_{i 1}\delta_{j 4}
.\end{align}
Thus, since addition is defined component wise, we have the anticommutator
\begin{align}
    \label{eq:anticommutator}
    ({\theta_1,\theta_2})_{ij} &= {(\theta_1)_{ij},(\theta_2)_{ij}} = \delta_{i_1}\delta_{j_4} + (-\delta_{i 1}\delta_{j 4}) = 0 \\
    \Rightarrow& \eqbox{\{\theta_1,\theta_2\} = 0}
.\end{align}


\prob{1.12}{Show that the Minkowski product $(x,y) = \va{x} \cdot \va{y} - x^{0}y^{0}$ of two 4-vectors $x$ and $y$ is an inner product obeying the rules (1.78,1.79,1.84).}

The three properties we check for this inner product are as follows:
\begin{subequations} 
\begin{eqnarray}
    \label{eq:prop-1}
    (f,g) = (g,f)^{*}
\end{eqnarray}
\begin{eqnarray}
    \label{eq:prop-2}
    (f,zg+wh) = z(f,g) + w(f,h)
\end{eqnarray}
\begin{eqnarray}
    \label{eq:prop-3}
    (f,g) = 0 \mbox{ for all } f \in V \Rightarrow g = 0
.\end{eqnarray}
\end{subequations}
For the following, we assume that $x,y$ are real 4-vectors.
That is, their components are real numbers.

From this assumption, \eref{prop-1} follows trivally
\begin{eqnarray}
    \label{eq:prop-1-mink}
    \eqbox{
    (x,y) = \va*{x} \cdot \va*{y} - x^{0}y^{0} = \va*{y} \cdot \va*{x} - y^{0}x^{0} = (y,x) 
    }
.\end{eqnarray}
Additionally, \eref{prop-2} is satisfied as follows:
\begin{align}
    \label{eq:prop-2-mink}
    (x,\alpha y + \beta z) &= \va*{x} \cdot (\alpha \va*{y} + \beta \va*{z}) - x^{0}(\alpha y^{0} + \beta z^{0}) \notag \\
                           &= \alpha \va*{x} \cdot \va*{y} + \beta \va*{x} \cdot \va*{z} - \alpha x^{0}y^{0} - \beta x^{0}z^{0} \notag \\
                           &= \alpha \left( \va*{x} \cdot \va*{y} - x^{0}y^{0} \right) + \beta \left( \va*{x} \cdot \va*{z} - x^{0} z^{0} \right) \notag \\
                           &= \eqbox{\alpha (x,y) + \beta (x,z)}
.\end{align}
Finally, we prove \eref{prop-3} as follows.
Suppose that $(x,y) = 0$ for all $x$ but $y \ne 0$.
Then, there is some component of $y$, say $y^{i}$ such that $y^{i} \ne 0$.
Obviously then, we could choose $x^{i} = 1$ and the rest of the components of $x$ to be zero, meaning that 
\begin{eqnarray}
    \label{eq:prod-3-mink}
    (x,y) = \begin{cases}
        y^{i} & \mbox{if } i \ne 0 \\
        y^{i} & \mbox{if } i = 0
    \end{cases}
.\end{eqnarray}
That is, it must be the case that $y \equiv 0$, otherwise there is always at least one choice of $x$ such that $(x,y)$ is nonzero, which contradicts the original assumption.


\prob{1.18}{Use the Gram-Schmidt method to find orthonormal linear combinations of the three vectors
\begin{eqnarray}
\label{eq:gram-schmidt-vectors}
    \va{s}_{1} = 
    \begin{pmatrix}
    1 \\
    0\\
    0
    \end{pmatrix},
    \quad
    \va{s}_{1} = 
    \begin{pmatrix}
    1 \\
    1\\
    0
    \end{pmatrix},
    \quad
    \va{s}_{3} = 
    \begin{pmatrix}
    1 \\
    1\\
    1
    \end{pmatrix}
.\end{eqnarray}
}

Notice that $\vu*{u}_{1} = \va*{s}_{1}$ is already normalized, so we can find an orthonormal vector to $\vu*{u}_{2}$ from $\va*{s}_{2}$ as
\begin{align}
    \label{eq:u2-vec}
    \va*{u}_{2} &= \vu*{s}_{2} - (\vu*{u}_{1},\va*{s}_{2})\vu*{u}_{1} \\
                &= \begin{pmatrix} 1 \\ 0 \\ 0\end{pmatrix} - \begin{pmatrix} 1 \\ 0 \\ 0\end{pmatrix} \\
                &= \begin{pmatrix} 0 \\ 1 \\ 0\end{pmatrix}
.\end{align}
Noting that $\va*{u}_{2}$ is normalized, we have $\vu*{u}_{2} = \va*{u}_{2}$.
Lastly, we can repeat the process for the third vector.
\begin{align}
    \label{eq:u3-vec}
\va*{u}_{3} &= \va*{s}_{3} - (\vu*{u}_{1},\va*{s}_{3})\vu*{u}_{1} - (\vu*{u}_{2},\va*{s}_{3})\vu*{u}_{2} \\
           & = \begin{pmatrix} 1 \\ 1 \\ 1\end{pmatrix} - \begin{pmatrix} 1 \\ 0 \\ 0\end{pmatrix} - \begin{pmatrix} 0 \\ 1 \\ 0\end{pmatrix} \\ 
           &= \begin{pmatrix}
    0 \\ 0 \\ 1
    \end{pmatrix}
.\end{align}
Again, $\va*{u}_{3} = \vu*{u}_{3}$ since $\va*{u}_{3}$ is already normalized.
Notice that these are just the standard basis vectors for a three dimensional vector space, so the orthonormal linear combinations of $\va*{s}_{1}$, $\va*{s}_{2}$, and $\va*{s}_{3}$ are
\begin{align}
    \label{orthonormal-vecs} 
    \eqbox{
    \vu*{u}_{1} = \va*{s}_{1}, \qquad
    \vu*{u}_{2} = \va*{s}_{1} - \va*{s}_{2}, \qquad
    \vu*{u}_{3} = \va*{s}_{1} - \va*{s}_{2} - \va*{s}_{3}
    }
.\end{align}


\prob{1.21}{Show that a linear operator $A$ that is represented by a hermitian matrix (1.167) in an orthonormal basis satisfies $(g,Af) = (Ag,f)$.}

Observe that
\begin{align}
    \label{eq:expand-inner-prod}
    (g,Af) = g^{\dagger}Af
,\end{align}
where $g,f$ are column vectors and $A$ is represented by a matrix.
Recalling that $(AB)^{\dagger} = B^{\dagger}A^{\dagger}$ and that $A^{\dagger} = A$ since $A$ is hermitian, \eref{expand-inner-prod} becomes
\begin{eqnarray}
    \label{eq:inner-prod-result}
    \eqbox{
    (g,Af) = (A^{\dagger}g)^{\dagger}f = (A g)^{\dagger} f = (Ag,f)
}
.\end{eqnarray}



\end{document}
